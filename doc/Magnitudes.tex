% % Simple tex document describing magnitude conversions...

\documentclass[11pt]{article}

\usepackage{graphics}
\usepackage{epsfig}
\usepackage[usenames]{color}
\usepackage{xspace}


\topmargin -1.5cm        % read Lamport p.163
\oddsidemargin -0.04cm   % read Lamport p.163
\evensidemargin -0.04cm  % same as oddsidemargin but for left-hand
\textwidth 16.59cm
\textheight 21.94cm
% \pagestyle{empty}       % Uncomment if don't want page numbers
\parskip 7.2pt           % sets spacing between paragraphs
% \renewcommand{\baselinestretch}{1.5} % Uncomment for 1.5 spacing between lines
\parindent 0pt           % sets leading space for paragraphs


% ********************* MY DEFINITIONS *****************************
\def\ni{\noindent}                                       %No indent%
\def\ls{\vskip 12.045pt}                            %One Line space%
\def\etal{et\thinspace al.\ }                               %et al.%
\def\etp{et\thinspace al.}    %et al., but with no space at the end%

\def\ojo{\fbox{\bf !$\odot$j$\odot$!}}      %Olho! Needs Correction%

\newcommand{\Ca}[1]{{\bf\color{red} #1}}
\newcommand{\Cb}[1]{{\bf\color{blue} #1}}
% \newcommand{\Cc}[1]{{\color{BurntOrange} #1}}
\newcommand{\Cc}[1]{{\bf\color{RawSienna} #1}}
\newcommand{\Cd}[1]{{\bf\color{OliveGreen} #1}}
% *******************************************************************

\begin{document}

\section{From spectrum to magnitudes}

\begin{equation}
m_{AB} = -2.5 \log{ {\int F_\nu T_\nu d\nu}\over{\int T_\nu d\nu} } - 48.6
\end{equation}

\ni but,

$$ F_\lambda d\lambda = F_\nu d\nu $$ and $$ \nu = \frac{c}{\lambda} $$

so:

\begin{equation}
m_{AB} = -2.5 \log { {\int F_\lambda T_\lambda d\lambda}\over{\int T_\lambda \frac{c}{\lambda^2} d\lambda} } - 48.6
\end{equation}

\begin{equation}
\label{eq:m_ab}
m_{A B} = -2.5 \log \int F_\lambda T_\lambda d\lambda + 2.5 \log \int T_\lambda \frac{c}{\lambda^2} d\lambda - 48.6
\end{equation}

\section{Error Propagation}

For error calculation matters, we rewrite equation \ref{eq:m_ab} as:
\begin{equation}
\label{eq:m_ab_sum}
m_{A B} = -2.5 \log \sum_\lambda F_\lambda T_\lambda \Delta \lambda + 2.5 \log \sum_\lambda T_\lambda \frac{c}{\lambda^2} \Delta\lambda - 48.6
\end{equation}

\ni in our case, we do not consider the error on $\lambda$ and the error on
$T_\lambda$, so the equation can be simplified to:

\begin{equation}
m_{A B}^\prime = -2.5 \log \sum_\lambda F_\lambda T_\lambda \Delta \lambda
\end{equation}

and, then, we to simplify the error calculations we rewrite it:
\begin{equation}
m_{A B}^\prime = -2.5 \log \sum_\lambda G(F_\lambda)
\end{equation}

\ni and divide the error calculation in two parts: one for $\sum_\lambda G(F_\lambda)$ and other
for the $\log$ of it. So, the first part is then:

\begin{equation}
\sigma^2(\sum_\lambda G(F_\lambda))
= \sum_\lambda \left( \frac{\partial F_\lambda T_\lambda \Delta\lambda}{\partial F_\lambda} \sigma(F_\lambda) \right)^2 \\
= \Delta\lambda^2 \sum_\lambda T^2_\lambda \sigma^2(F_\lambda)
\end{equation}

\newpage{}

and the second part will be:

\begin{equation}
\sigma^2(m_{AB}) = \left|\frac{\partial m_{AB}}{\partial \sum_\lambda G}\right|^2\ \sigma^2 \left(\sum_\lambda G \right)
\end{equation}

which gives us:

\begin{equation}
\sigma(m_{AB}) = \frac{2.5}{\ln 10} \sqrt{\frac{ \sigma^2(\sum_\lambda G) }{\left(\sum_\lambda F_\lambda T_\lambda \Delta_\lambda\right)^2}}
\end{equation}

\begin{equation}
\sigma(m_{AB}) = \frac{2.5}{\ln 10} \frac{\sqrt{\sum_\lambda T^2_\lambda \Delta^2_\lambda \sigma^2(F_\lambda)}}
				                         {\sum_\lambda F_\lambda T_\lambda \Delta_\lambda}
\end{equation}

if we consider $\Delta \lambda$ constant on the filter window, we can simplify the equation to:

\begin{equation}
\sigma(m_{AB}) = \frac{2.5}{\ln 10} \frac{\sqrt{\sum_\lambda T^2_\lambda\sigma^2(F_\lambda)}}
											{   {\sum_\lambda F_\lambda T_\lambda} }
\end{equation}

One can see that the denominator is N times the weighted (by $T_l$) mean flux.
The numerator, on the other hand, is $\sqrt{N}$ times the weighted mean RMS
error, so that in the end the error in magnitude is $\frac{1}{\sqrt{N}}$ times
the typical relative error $\frac{\sigma_l}{F_l}$.


\end{document}
